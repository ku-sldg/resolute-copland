% This is samplepaper.tex, a sample chapter demonstrating the
% LLNCS macro package for Springer Computer Science proceedings;
% Version 2.20 of 2017/10/04
%
\documentclass[runningheads]{llncs}
%
\usepackage{graphicx}
% Used for displaying a sample figure. If possible, figure files should
% be included in EPS format.
%\usepackage{tikz}
%\usetikzlibrary{arrows}
\usepackage{verbatim}
\usepackage{algorithm}
\usepackage[noend]{algpseudocode}
\usepackage{amssymb}
\usepackage{amsfonts}
\usepackage{amsmath}
\let\proof\relax\let\endproof\relax
\usepackage{amsthm}
% \usepackage{graphicx}
%\usepackage[all]{xy}
\usepackage{semantic}
\usepackage{array}
\usepackage{enumitem}
%\usepackage{cite}
\usepackage[numbers,sectionbib]{natbib}
\usepackage{wrapfig}
\theoremstyle{definition}
\renewcommand{\qedsymbol}{\hfill\ensuremath{\blacksquare}}
%\newtheorem{definition}{Definition}[section]
% If you use the hyperref package, please uncomment the following line
% to display URLs in blue roman font according to Springer's eBook style:
% \renewcommand\UrlFont{\color{blue}\rmfamily}
\usepackage[breaklinks=true]{hyperref}
\usepackage{breakcites}
\renewcommand\UrlFont{\color{blue}\rmfamily}

\input{lib/coq-listings}

\begin{document}
%
\title{RESOLUTE and Copland%\thanks{Supported by organization x.}
}
%
%\titlerunning{Abbreviated paper title}
% If the paper title is too long for the running head, you can set
% an abbreviated paper title here
%
\author{Uk'taad B'mal}
%
\authorrunning{Uk'taad B'mal}
% First names are abbreviated in the running head.
% If there are more than two authors, 'et al.' is used.
%
\institute{Institute for Information Sciences \\ The
  University of Kansas \\ Lawrence, KS 66045 \\
  \email{palexand@ku.edu}}
%
\maketitle              % typeset the header of the contribution
%
\begin{abstract}
The abstract should briefly summarize the contents of the paper in
15--250 words.

\keywords{\textsf{Copland}  \and \textsf{RESOLUTE} \and formal semantics.}
\end{abstract}
%
%
%

\section{Introduction}

\reservestyle{\command}{\textsf}
\command{let,in,and,or}

\begin{definition}[Compilation]\\
  \comp{t}{L} defines compiling \textsf{RESOLUTE} term $t$ into a
  \textsf{Copland} protocol in the context of AM library L.
\end{definition}

\begin{definition}[Strategy] \\
  \comp{strategy $id$ "copland.\emph{L}" p}{} = \eval{p}{L}
\end{definition}

Specify the analysis strategy is using Copland with AM Library L to
map interpreted RESOLUTE functions to their implementing ASPs.  The AM
Library provides implementations for abstract analysis activities.


%%% Local Variables: ***
%%% mode:latex ***
%%% TeX-master: "resolute-copland.tex"  ***
%%% End: ***

\section{Claims and Rules}

Inference rules taken from~\citet{gacek2014resolute}.

\inference{\Gamma |- G_1 & \Gamma |- G_2}{\Gamma |- G_1 \wedge G_2}[$\wedge$]

\inference{\Gamma |- G_i}{\Gamma |- G_1 \vee G_2}[$\vee_{i=1,2}$]

\begin{definition}[Conjunction and Disjunction] \\
  \comp{$t_1$ \<and> $t_2$}{L} = \comp{$t_1$}{L} $\mathsf{+\!<\!+}$
  \comp{$t_2$}{L} $->$ aprAnd \\
  \comp{$t_1$ \<or> $t_2$}{L} = \comp{$t_1$}{L} $\mathsf{+\!<\!+}$
  \comp{$t_2$}{L} $->$ aprOr
\end{definition}

Compile the two binary operator terms using the same AM Library into a
sequential Copland ordering.  Appraise the result to determine if the
conjunction or disjunction holds.  Attestation gathers evidence,
appraisal determines if it adheres to policy.

\inference{\Gamma,G_1|-G_2}{\Gamma|-G_1=> G_2}[$=>$]

\begin{definition}[Implication]
  
\end{definition}

\begin{minipage}{0.45\linewidth}
 \inference{\Gamma|-G(t_1) & \ldots & \Gamma|-G(t_n)}{\Gamma |- \forall
  x:\alpha \cdot G(x)}[$\forall$]
\end{minipage}
\begin{minipage}{0.45\linewidth}
  \inference{\Gamma|-G(t_i)}{\Gamma |- \exists x:\alpha
    \cdot G(x)}[$\exists_{i=1\ldots n}$]
\end{minipage}

where $\alpha=\{t_1,\ldots,t_n\}$.

Resolute quantification is defined over finite sets $\alpha$. Thus,
tests of universal and existential quantification statements
generalize $\wedge$ and $\vee$ to arbitrarily many terms.

\begin{definition}[Universal and Existential]\\
  \comp{$\forall x:\alpha \cdot G(x)$}{L} = \comp{$t_1$}{L} $\mathsf{+\!<\!+}$
  $\ldots$ $\mathsf{+\!<\!+}$\comp{$t_n$}{L} $->$ aprAnd \\
  \comp{$\exists x:\alpha \cdot G(x)$}{L} = \comp{$t_1$}{L} $\mathsf{+\!<\!+}$
  $\ldots$ $\mathsf{+\!<\!+}$\comp{$t_n$}{L} $->$ aprOr
\end{definition}

\inference{\Gamma|-G(\bar{t})}{\Gamma|-A(\bar{t})}[Backchain]

where $\forall \bar{x}\cdot A(\bar{x})\in\Gamma$

\begin{definition}[Backchain]

\end{definition}

\inference{e\Downarrow\bot}{\Gamma,\langle e\rangle|-G}[eval-L]
\inference{e\Downarrow\top}{\Gamma|-\langle e\rangle}[eval-R]

\begin{definition}[Eval]
  
\end{definition}

\section{Computations}

\begin{definition}[Uninterpreted Function]\\
\comp{$f(p)$}{L} = ASP L(f) p
\end{definition}

Compile a primitive, uninterpreted function into an ASP call using
library L.

%
% 
% ---- Bibliography ----
%
% BibTeX users should specify bibliography style 'splncs04'.
% References will then be sorted and formatted in the correct style.
%
%\bibliographystyle{splncs04}
\bibliographystyle{splncsnat}
%\bibliography{sldg}
\bibliography{bib/sldg}
%
\end{document}
